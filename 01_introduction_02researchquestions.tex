\section{Research Questions}

\begin{enumerate}
    \item The RoBERTa model can be used for sentiment analysis on social media posts. How can it be used to compare and contrast the two elections between two countries, the Philippines and the USA, in terms of social media activity?
    \begin{enumerate}
        \item What are the sentiments and texts RoBERTa captured from Philippine social media users as opposed to US social media users?
        \item How do the different themes, frequencies, and sentiments of keywords and phrases expressed by users on X (formerly Twitter) indicate their support for the candidates and a candidate's electoral win?
        \item What are the similarities and differences, if there are any, between the results and findings from the RoBERTa model for the Philippines and United States?
        \item How can social media presence throughout the two elections (Marcos Jr. vs. Robredo in the 2022 Philippine elections; and Trump vs. Harris in the 2024 US elections) be visualized to show concise and understandable information to the general public?
    \end{enumerate}
\end{enumerate}