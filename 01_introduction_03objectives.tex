\section{Research Objectives}
\begin{enumerate}
    \item Agent-Based Modelling can be used for creating simulations of various social phenomena. For this paper, the researchers intend to test its ability to simulate not only candidate-user interactions but also user-user interactions in social media during Philippine election seasons.
    \begin{enumerate}
        \item Through XLM-R, perform sentiment analysis on social media posts about both the 2019 and 2022 Philippine elections and create a list of parameters with which Agents and interactions in the simulation are defined.
        \item Through our baseline model, perform an Agent-Based Simulation to evaluate the efficacy of the model to predict winning candidates of the 2025 Philippine senatorial elections, using information diffusion to represent the amount of social interactions done in the simulation with respect to a given candidate.
        \item After verifying the accuracy of the Agent-Based model, perform a simulation to predict potential winning candidates of the upcoming 2028 Philippine presidential elections, given a hypothetical candidate’s online presence and sentiments by users surrounding them.
    \end{enumerate}
\end{enumerate}