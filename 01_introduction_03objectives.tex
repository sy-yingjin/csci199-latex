\section{Research Objectives}
\begin{enumerate}
    \item An Agent-Based Model will be created using sentiments extracted from posts on social media platform X, for the purpose of simulating not only candidate-user interactions but also user-user interactions in election campaigning periods. 
    \begin{enumerate}
        \item Through XLM-R, perform sentiment analysis on social media posts about both the 2019 and 2022 Philippine elections and create a list of parameters with which Agents and interactions in the simulation are defined.
        \item Through our baseline model, perform an Agent-Based Simulation to evaluate the efficacy of the model to predict winning candidates of the 2025 Philippine senatorial elections, using information diffusion to represent the amount of social interactions done in the simulation with respect to a given candidate. Then, after verifying the accuracy of the Agent-Based model, perform a simulation to predict potential winning candidates of the upcoming 2028 Philippine presidential elections, as of the political landscape in the current year of writing (2025).
        \item Compare the results of the model with actual election results to determine if there is a statistically significant difference between them, i.e. through t-testing simulated results with actual election results.
    \end{enumerate}
\end{enumerate}