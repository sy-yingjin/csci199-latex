\chapter{METHODOLOGY}
The methodology will follow the Figure \ref{fig:Methodology}.

\begin{figure}[h]
    \centering
    \includegraphics[width=1\textwidth]{Figures/methodology_flowchart.png}
    \caption{Methodology Flowchart}
    \label{fig:Methodology}
\end{figure}

\clearpage

\section{Data Collection}
There are publicly available datasets for the 2024 US Presidential elections; however, for the 2022 Philippine Presidential Elections, the researchers will have to use the X (formerly Twitter) Application Programming Interface (API) Version 2 for academic access. Keywords will filter out tweets not included in the study’s scope.
\newline

\begin{table}[h]
    \centering
    \begin{tabularx}{\textwidth}{X|X|X|X}
        \textbf{Presidential Election Year} & \textbf{Intention to Run Announcement} & \textbf{Dates within the Dataset (Inclusive)} & \textbf{Key Terms}\\
        \hline\hline
        \multirow{2}{*}{2022}& Marcos and Duterte: October 5, 2021 & \multirow{2}{*}{May 7, 2022} & \multirow{2}{4cm}{Marcos, Duterte, Robredo, Kiko, Pangilinan, BBM, Leni, DU30, Uniteam, Bongbong} \\
        & Robredo and Pangilinan: October 7, 2021 & & \\
    \end{tabularx}
    \caption{Table for Philippine Dataset Ranges}
\end{table}

\begin{table}[h]
    \centering
    \begin{tabularx}{\textwidth}{X|X|X|X}
        \textbf{Presidential Election Year} & \textbf{Intention to Run Announcement} & \textbf{Dates within the Dataset (Inclusive)} & \textbf{Key Terms}\\
        \hline\hline
        \multirow{2}{*}{2024}& Trump and Vance: November 15, 2022 & \multirow{2}{*}{November 6, 2024} & \multirow{2}{4cm}{Harris, Walz, Trump, Vance, Kamala, JD Vance} \\
        & Harris and Walz: July 21, 2024 & & \\
    \end{tabularx}
    \caption{Table for United States Dataset Ranges}
\end{table}


Each collected tweet will be saved in the \texttt{.csv} file, separated, with each row marked by which country it belongs to, because it will be utilized for text classification.

\section{Data Preprocessing}
Before a group of datasets is fed into a tokenizer, they will undergo text processing.  Stop words such as \emph{“the,” “a,” “is”}, etc., will not be removed as they might be considered for the full context of a sentence when performing byte-tokenization. The following steps to preprocess the text will be as follows:

\begin{enumerate}
    \item Omitting a tweet from the dataset if it is not in English, Tagalog, or a mix of them. The languages of the tweets will be detected by the Python library \texttt{langdetect}.
    \item Removing punctuation marks that have no significance for sentiment analysis.
    \item Replacing emojis with special tags.
    \item Removing unnecessary emojis or replacing emojis with special tags describing them if it is necessary for sentiment analysis.
    \item Lowercasing the text.
    \item Handling links and email addresses by replacing them with a placeholder.
    \item Removing whitespaces and replacing multiple spaces with a single space.
    \item Adding paddings to equalize the length of sentences.
\end{enumerate}

The preprocessed dataset will be placed in a new \texttt{.csv} file.

\section{Text Classification and Visualization}
After the preprocessing, the dataset will be fed into the following models: the \textsc{XLM-RoBERTa model} for contextualized embedding and the TF-IDF model for determining frequent keywords. Its language-agnostic approach makes it easier to implement as there is no need to identify if certain texts are in English or Tagalog. The Twitter-XLM-RoBERTa model will be used for tokenization and embeddings of tweets. Once the preprocessed dataset is fed into the model, the resulting tokens will be used to determine frequent keywords via TF-IDF and semantic similarities on the embedding model. Once the embeddings are generated, they will be compared for semantic similarity via K-Means clustering. Since the model has a feature that helps determine if a tweet’s sentiment is positive, negative, or neutral, it will become a color indicator in the visualization of clusters to determine a cluster’s sentiment and visualize echo chambers. This will be crucial in comparing and contrasting the social media presence and activity of each candidate.

Meanwhile, the tokenized texts will also go to the TF-IDF model to determine the frequency of words. The frequency of the words is sorted by how frequently they are used in a certain post or comment. The top 30 keywords per candidate will be used to compare and contrast with other candidates and the social media spheres of the two countries.\newline

\begin{figure}[h]
    \centering
    \includegraphics[width=0.65\textwidth]{Figures/methodology_framework-for-comparing.png}
    \caption{Framework for Comparing and Contrasting Sentiments}
    \label{fig:Framework-for-sentiments}
\end{figure}