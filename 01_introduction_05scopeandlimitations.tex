\section{Scope and Limitations of the Study}
This paper consists of two case studies to provide a rich analysis of social media data extracted from presidential election periods. First, a case study analysis comparing the recent 2022 Philippine presidential election, with leading candidates Ferdinand “Bongbong” Marcos Jr. and Maria Leonor "Leni" Robredo, against the recent 2024 US presidential Election, with leading candidates Donald John Trump and Kamala Devi Harris. Second, a comparison between the 2016 Philippine presidential election, with leading candidates Rodrigo Roa Duterte and Manuel "Mar" Araneta Roxas II, and the 2020 US presidential elections, with leading candidates Donald John Trump and Joseph Robinette “Joe” Biden Jr. To add to the discussion, social media data about each president’s respective vice presidential running mate candidate will be considered as well. The paper aims to analyze how the respective campaign periods of each candidate (and their running mates) were reflected on different social media spaces. This is so that the researchers can determine the effectiveness of social media platforms as indicators of electoral wins, and determine whether electoral results can be foreseen based on online traction and popularity.

This paper will focus only on social media platforms such as Facebook, Twitter, TikTok, in data collection due to their popularity within the United States and the Philippines, which significantly increases the chances of attaining a statistically large dataset in comparison to other social media platforms. Only posts made prior to the actual election days will be collected, as the goal is to compare pre-election social media data to post-election results. Thus, posts made during the election days and beyond will be excluded from the study. It is also worth mentioning that, for all elections under the scope of this study, only the winner and first-runner up will be considered as collecting enough data on all candidates might not be feasible.\cite{RRL_Macrohon-2022} Finally, “posts” include any and all tweets made by the general public on the selected candidates alongside any posts made by the candidates themselves.

Finally, attached to the names of some candidates are certain criminal cases. The Marcos Family was responsible for a series of atrocities and human rights violations in the 1970s, and Donald Trump is currently facing multiple criminal cases.\cite{SaL_AlJazeera-2024,SaL_AmnInt-2022} The paper will not explore such topics in-depth as they are outside of the scope and focus of the study; however, these may be touched upon briefly if it is a popular discussion point among users in the data collected.