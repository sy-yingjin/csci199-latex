\section{Scope and Limitations of the Study}
This paper consists of two case studies to provide a rich analysis of social media data extracted from four presidential election periods. To add to the discussion, social media data about each president’s respective vice presidential running mate candidate will be considered as well. As such, first, a case study analysis comparing the recent 2022 Philippine presidential election, with leading candidates Marcos and Robredo, and their respective running mates Duterte and Pangilinan, against the recent 2024 US presidential election, with leading candidates Trump and Harris, and their respective running mates Vance and Walz. Second, a comparison between the 2016 Philippine presidential election, with leading candidates Duterte and Roxas II, and their respective running mates Cayetano and Robredo, and the 2020 US presidential elections, with leading candidates Trump and Biden, and their respective running mates Pence and Harris. The paper aims to analyze how the respective campaign periods of each candidate (and their running mates) were reflected on different social media spaces. This is to determine the effectiveness of social media platforms as indicators of electoral wins and determine whether electoral results can be foreseen based on online traction and popularity.

This paper will focus only on the social media platforms Facebook, Twitter, TikTok, in data collection due to their popularity within the United States and the Philippines. Only posts made after the announcement of a candidate’s intention to run for president and prior to the actual election days will be collected, as the goal is to compare pre-election social media data to post-election results. As such, with reference to the dates of when each leading candidates announced their intention to run, the paper will consider posts made after November 21, 2015 for Duterte and July 31, 2015 for Roxas in the 2016 Philippine presidential election, October 7, 2021 for Robredo and October 5, 2021 for Marcos in the 2022 Philippine presidential election, April 25, 2019 for Biden and January 21, 2017 for Trump in the 2020 US presidential election, and July 21, 2024 for Harris and November 15, 2022 for Trump in the 2024 US presidential election \cite{SaL_Ranada-2015, SaL_Cupin-2015, SaL_Lalu-2021, SaL_Buan-2021,SaL_Beaumont-2019,SaL_Gold-2017,SaL_Viner-2024, SaL_Orr-2022}. Posts made 2 days before the election dates and beyond will be excluded from the study. It is also worth mentioning that, for all elections under the scope of this study, only the winner and first-runner up will be considered as collecting enough data on all candidates might not be feasible \cite{RRL_Macrohon-2022}. Finally, \emph{“posts”} include any and all publicly available posts made by the general public on the selected candidates alongside any posts made by the candidates themselves; however, data on the users who made the posts themselves, such as gender or location, will not be considered due to the lack of availability.

Finally, attached to the names of some candidates are certain criminal cases. The Marcos Family was responsible for a series of atrocities and human rights violations in the 1970s, and Trump is currently facing multiple criminal cases \cite{SaL_AlJazeera-2024,SaL_AmnInt-2022}. The paper will not explore such topics in-depth as they are outside of the scope and focus of the study; however, these may be touched upon briefly if it is a popular discussion point among users in the data collected.