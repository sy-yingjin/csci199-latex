\section{Scope and Limitations of the Study}
The scope of the study is to create a virtual environment that simulates the Philippine social media during an election season, with both users and candidates, through ABM.

Data collection will be limited to three Philippine election seasons for training and evaluation of the model: namely, the 2019 midterm and 2022 presidential elections for training purposes, and the 2025 midterm elections for validation. Scraping of social media posts will be restricted within X due to the short character limit of 280 for each post, which makes performing sentiment analysis more manageable than on other platforms. The range of dates for posts collected depends on the last day of filing a Certificate of Candidacy (COC), and the last day before the election day proper. Posts made on the election days themselves will not be included in our dataset.

As for the number of candidates, all presidential candidates for the 2022 elections will be the main agents; for senatorial elections, on the other hand, only posts about the top 20 candidates will be collected from their official pages. This is so that the model is still able to capture the contribution of those outside the winning 12 candidates to online public discourse on which candidates to vote for.