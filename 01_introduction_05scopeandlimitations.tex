\section{Scope and Limitations of the Study}
This paper is a case study on two presidential elections that have most prominently used social media as part of their campaign strategies, with data extracted each from the Philippines and the United States. To add to the discussion, social media data about each president’s respective vice presidential running mate candidate will be considered as well. As such, this is a case study analysis comparing the recent 2022 Philippine presidential election, with leading candidates Marcos-Duterte and Robredo-Pangilinan against the recent 2024 US presidential election, with leading candidates Trump-Vance and Harris-Walz. The paper aims to analyze how the respective campaign periods of each candidate (and their running mates) were reflected on different social media spaces. This is to determine the effectiveness of social media platforms as indicators of electoral wins and determine whether electoral results can be foreseen based on online traction and popularity.

This paper will focus only on the social media platform X (formerly Twitter) in data collection due to their popularity within the United States and the Philippines. Only posts made after the announcement of a candidate’s intention to run for president and prior to the actual election days will be collected, as the goal is to compare pre-election social media data to post-election results. It is also worth mentioning that, for all elections under the scope of this study, only the winner and first-runner up will be considered as collecting enough data on all candidates might not be feasible \cite{RRL_Macrohon-2022}. As such, with reference to the dates of when each leading candidates announced their intention to run, the paper will only consider posts made after October 7, 2021 for Robredo and October 5, 2021 for Marcos in the 2022 Philippine presidential election, and July 21, 2024 for Harris and November 15, 2022 for Trump in the 2024 US presidential election \cite{ SaL_Lalu-2021, SaL_Buan-2021, SaL_Viner-2024, SaL_Orr-2022}. Posts made 2 days before the election dates and beyond will be excluded from the study. \emph{“Posts”} include any and all publicly available posts made by the general public on the selected candidates salongside any posts made by the candidates themselves; however, data on the users who made the posts themselves, such as gender or location, will not be considered due to the lack of availability.

Finally, attached to the names of some candidates are certain criminal cases. The Marcos Family was responsible for a series of atrocities and human rights violations in the 1970s, and Trump is currently facing multiple criminal cases \cite{SaL_AlJazeera-2024,SaL_AmnInt-2022}. The paper will not explore such topics in-depth as they are outside of the scope and focus of the study; however, these may be touched upon briefly if it is a popular discussion point among users in the data collected.