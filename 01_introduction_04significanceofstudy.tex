\section{Significance of the Study}
This paper aims to contribute to the field of social computing by analyzing the behavior of users and political candidates in online spaces during election periods, especially in the Philippine context. Using sentiment analysis and other NLP techniques, the research aims to gather statistically large amounts of social media data to represent the interaction between candidates and the general public, and between users as well, who might be potential voters. Then it utilizes models that process such data and draw insights from the sentiments expressed in social media posts by candidates and their audiences, particularly during presidential and midterm elections.

The analytics and findings of this study will benefit the following: Philippine political campaign teams in strategizing social media posts to build towards this particular sentiment to gather shifts in voting preference, the general public in understanding the information diffusion of Philippine social media during elections, and political analysts in grasping a wider view of social media networks during an election season.