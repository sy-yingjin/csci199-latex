\section{Significance of the Study}
This paper aims to contribute to the field of social computing by analyzing the behavior of users in online spaces during election periods. By way of sentiment analysis and other NLP techniques, the research aims to gather statistically large amounts of social media data that represent the general public, and after which makes use of models that process such data and draw insights from the sentiments behind social media posts concerned with presidential elections. 

The analytics and findings of this study will benefit the general public's knowledge of how social media runs in both Philippine and American contexts, especially during the election seasons. These findings will also provide them with more context between the spheres of social media in terms of various political ideologies that the said candidates have. The researchers hope to impart realizations on how social media, despite shaping public opinions, might not be the definitive factor behind electoral wins.