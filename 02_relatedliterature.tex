\chapter{REVIEW OF RELATED LITERATURE}
The review of related literature and studies section of the research examines existing studies that tackle the following: the importance of sentimental analysis, especially in the context of elections; the definition of the BERT model and how it is an effective model in classifying sentiments of the social media posts; how the social media has become a critical tool for political engagement building the general public’s sentiment; and background on the Philippine and US presidential elections to be analyzed. 

The sentiment analysis subsection discusses its definition and how it is essential in analyzing social media engagement. Then, it discusses the tools used for analysis, especially the transformation model \texttt{Bidirectional Encoder Representations from Transformers (BERT)}—its architecture, its variations, including RoBERTa, and how effective it is for sentiment analysis using various studies as evidence. Lastly, the Elections and Social Media subsection is for discussion on how social media shapes the general public, especially in the context of two countries the Philippines and the United States.

\section{Sentiment Analysis for Social Media and Elections}
According to Liu [2012], the study of people's views, sentiments, assessments, appraisals, attitudes, and emotions about goods, services, organizations, people, problems, events, subjects, and their characteristics is commonly referred to as sentiment analysis or opinion mining \cite{RRL_Liu-2012}. With the explosive growth of social media, it has become a hotspot of opinions, shaping our decisions, especially in an important political event like elections. In the field of social computing, election seasons are one of the widely researched topics, especially on how the interaction in social media affects society in terms of making decisions on who to vote for. 

There are examples of studies using sentiment analysis to analyze social media activity. In a study by Macrohon, et al. [2022] and Demillo, et al. [2023], they used the \texttt{Naïve Bayes classifier}, a probabilistic learning method, to determine the probability of a tweet belonging to the best class—applicable in determining the polarity of a post \cite{RRL_Macrohon-2022,RRL_Demillo-2025}. Then, previous studies showed the usage of bidirectional encoder representation from transformers (BERT) models, modified to handle emojis and Tagalog language tweets. Aquino, et al [2025] introduced the emotion-infused \texttt{BERT-GCN model} for sentiment analysis, which includes emoji semantics into the models, treating them as sentiment representation  \cite{RRL_Aquino-2025}; meanwhile, Cruz, et al. [2022] used the \texttt{RoBERTa-tagalog-cased model} to get the vectorized version of Tagalog embeddings, essential to map echo chambers on Twitter via \texttt{K-Means modeling} \cite{RRL_Cruz-2022}. Lastly, the \texttt{Support Vector Machines (SVM) Classifier model} was used by Demillo, et al. [2023] to handle binary classification of data, classifying them as either a negative or positive sentiment \cite{RRL_Demillo-2025}.

For reasons discussed more in-depth in the next section, BERT was chosen for the study’s methodology given its capabilities of performing nuanced analyses and classification of social media posts.

\subsection{BERT}
Recent developments in devising models for NLP tasks have ensured that models are updated to be more context-aware, being able to provide a more holistic and nuanced analysis of certain texts. One such model is BERT, short for Bidirectional Encoder Representations from Transformers (referred to henceforth as BERT). Developed by the Google AI Language Laboratory, the main advantage provided by BERT is its ability to analyse text in a bidirectional manner, as opposed to more traditional machine learning models, such as GPT, which only analyse text left-to-right or vice versa. Bidirectional analyses of text ensures that BERT is able to capture not only the sentiments of text, but do so in such a manner that the model is able to detect certain nuances, such as sarcasm or irony. BERT is primarily pre-trained in two phases: first, using a large dataset of unlabeled data, then second; a smaller set of labeled data, usually for fine-tuning the BERT model according to some NLP task. One such NLP task is Sentiment Analysis. An improved variant of BERT, RoBERTa (short for Robustly Optimized BERT Pre-training Approach), was then developed by the Facebook AI research team. It uses longer training times and larger batch sizes in training; and as a result, is able to outperform BERT on the same NLP tasks \cite{RRL_Koroteev-2021}. 

As BERT and RoBERTa have seen usage in analysing large datasets of text, it is able to aid in research on social media, a rapidly evolving form of text widely different from more traditional text formats such as novels. This is mainly due to the widespread usage of informal language, abbreviations, and emojis, among other elements, which can be challenging to understand without the proper context. Nevertheless, Kumar and Sadanandam were able to use BERT and RoBERTa to classify a large dataset of some 8,225 tweets related to the Coronavirus into three general sentiments: positive, neutral, and negative. Both BERT and RoBERTa were able to perform sentiment analysis across the entire dataset, achieving high accuracies (at least 88\%), precision (at least 0.88), recall (at least 0.74 but can go as high as 0.91), and F1-score (at least 0.78 but can go as high as 0.90) \cite{RRL_Pranay-Kumar-2023}.

\section{Social Media Use and the Elections}
\subsection{Candidate Activity}
The 2016 Philippine presidential election is widely considered the first “social media election” in the Philippines \cite{RRL_Sinpeng-2020}. The two Philippine presidential elections, 2016 and 2022, are undoubtedly linked as they involve something other than the rise of the Marcos-Duterte alliance: the prominence of social media as a means to bolster their presence in people’s lives and boost their popularity.

Despite having the most engagement, Duterte’s online presence during his presidential campaign is nothing short of lackluster and underwhelming \cite{RRL_Sinpeng-2020}.

On the other hand, Marcos Jr’s campaign has been well-established and maintained in the years leading up to his campaign \cite{RRL_Mendoza-2022}. His pitch throughout the campaign calls for national unity, featuring the glorification of his father’s legacy. In Rappler’s three-part study on “networked propaganda” back in 2019, there was a rise in many pro-Marcos pages and channels on different social media platforms, notably on TikTok. There was less activity from Marcos Jr. himself, however, those channels were particularly full of pro-Marcos content \cite{RRL_Mendoza-2022}.

\subsection{Public Opinion}
Duterte’s successful campaign can be attributed to his aggressive supporters– most of whom are vocal online and active offline. As observed by Sinpeng, et al. [2020], despite Duterte’s unprofessional online presence, his supporters are committed and constantly rallied to his defense against the criticism of other candidates \cite{RRL_Sinpeng-2020}. There are also prospects of the heavy involvement of informal actors like paid trolls and influencers as having major roles in mobilizing (and agitating) digital communities, which helped spread his popularity \cite{RRL_Sinpeng-2020}.

“The recent election has been the most social media-active and engaging campaign in the country’s democratic history.” said Ampon, et al. [2023] in a paper analyzing the political message strategies of Marcos and Robredo.  In the 2022 Philippine presidential race, both leading candidates (Marcos Jr. and Robredo) have taken great leads on social platforms like Facebook and X, respectively \cite{RRL_Ampon-2023}. Electoral campaigns are aimed at spreading awareness about the candidate’s identity and, over the years, Marcos Jr. has amassed a large number of supporters on TikTok based on the top 4 trending hashtags related to him: \texttt{\#bongbongmarcos} (3.4 billion views), \texttt{\#bbmsara2022} (2.3 billion views), \texttt{\#uniteam} (2.5 billion views) and \texttt{\#bbm2022} (2 billion views) \cite{RRL_Mendoza-2022}.

\section{Elections Background}
\subsection{Philippine Elections (2016, 2022)}
In his study on the 2016 Philippine presidential elections, after analyzing pre-election surveys, the candidates' campaign strategies and their advocacies, and their supporters' age demographic and news tracking, Holmes [2016] observed that the elections in the Philippines are political clan-dominated, \\personality-oriented, and media driven\cite{RRL_Holmes-2016}.

Rodrigo Duterte's victory in the election was believable to the public and was attributed to: the clarity of his campaign slogan, his significant support from a geographic area, and how he criticized and questioned the character and competence of his fellow candidates. However, one of the most significant observations in the study is the importance of the media, which is updated in real-time and where voter preference was shaped and reformed by Duterte’s critiques and ‘bashing’\cite{RRL_Holmes-2016}.

Following Duterte’s term in office, Ferdinand ‘Bongbong’ Marcos Jr.’s electoral win has caused much uproar among the nation’s scholars. The general resurgence of the Marcos Clan in politics can be attributed to 3 factors: (1) the people’s nostalgia of the Marcos era, (2) Duterte’s political influence, and (3) the Marcos’ years-long digital disinformation campaign on social media\cite{RRL_Pernia-2025}. \\Duterte’s consequent influence on Marcos’ resurgence cannot be dismissed, as signs have pointed out that Duterte’s indirect endorsement led to his win\cite{RRL_Dulay-2023}.

\subsection{US Elections (2020, 2024)}
At the cusp of the COVID-19 pandemic, the 2020 US presidential election had taken a major hit– particularly for one of its leading candidates, Donald Trump, whose vote share is largely affected by COVID-19-related cases \cite{RRL_Baccini-2021}. It is likely that Trump was viewed negatively for how he handled the pandemic as the most affected counties and states are ones without stay-at-home orders, in swing states, or states that Trump won in 2016. This mismanagement is what likely led to changes in voter preferences and Joe Biden’s eventual electoral win.

The 2024 US presidential election was predicted to be one of the most competitive in modern history with a tight competition between candidates Donald Trump and Kamala Harris, the new face of the Democratic Party \cite{RRL_Setiawan-2025}. In the end, Trump had managed to win the electoral race \cite{RRL_Setiawan-2025}.

\section{Data Collection Methods}
In the Philippine context, as Filipino is a low-resource language, a lack of Filipino datasets of tweets has been a pressing issue from previous studies \cite{RRL_Aquino-2025}. In addition, in the American context, because of the recency of the 2024 US Presidential Elections, established public datasets are sparse. Thus, the research had to collect data through APIs and open-sourced scrapers. \texttt{X API v2} was used to retrieve relevant tweets from X (formerly Twitter) then put through a series of Python codes \cite{RRL_Ancheta-2020}.

Ways of extracting posts from Facebook and TikTok are different from ways of extracting tweets from X (formerly Twitter) because of the unique nature of their postings, which, for example, in the case of TikTok, consist of only images or videos. In posts from Facebook, in the study from Grujic, et al. [2014], they employ the Facebook Graph API, an HTML-based API to access information from Facebook \cite{RRL_Grujic-2014}. It utilized PHP, HTML, and jQuery to query data, post new stories, upload photos, and more. In the study by Alashri, et al. [2016], they used Python codes to extract Facebook posts and comments from the pages of presidential candFdates for the 2016 US Presidential Elections \cite{RRL_Alashri-2016}.

Meanwhile, in extracting data from TikTok, Vassey, et al. [2022] gathered data related to little cigar and cigarillo products by scraping public posts with hashtags containing high engagements \cite{RRL_Vassey-2022}. Researchers then developed a codebook to analyze themes within the videos and captions, establishing interrater reliability through subsample coding. Similar methods are also used by Abbas, et al. [2022] wherein they used the top three hashtags (\texttt{\#SaveSheikhJarrah}, \texttt{\#SavePalestine}, \texttt{\#FreePalestine}) to search TikTok video content related to youth activism in the Israel-Palestine conflict, however, it also used to find videos via non-random sampling method to code frame by frame to see what is the message of the video \cite{RRL_Abbas-2022}. Lastly, Cheng and Li [2023] converted the audio of news videos in TikTok to text via a Google API to train the text into a sentiment classifier \cite{RRL_Cheng-2024}. Then, they took images of them every second to calculate the second-person view ratio, essential for studying the prevalence of that person of view in every news video. Then, they took every frame-per-second of a video to calculate the second-person view ratio, essential for studying the prevalence of that viewpoint.

From the existing studies mentioned, although these methods are possible, it still poses a challenge for the researchers to collect social media data due to the social media platforms’ dynamic website structure, limited APIs, rate limits, and potential data noise that might be collected.