
\section{Context of Study}
In recent decades, social media has had a major role as a platform through which political ideologies are spread and political discussions occur. This is especially apparent when observing the flow of recent elections in certain countries. In the Philippines, the 2016 Philippine presidential election is widely considered the first “social media election” in the Philippines, mainly due to how its winner, \textsc{Rodrigo Roa Duterte}, was able to utilize social media to establish a controversial image which mobilized his wide follower-base to rally in support of him, both online and offline \cite{RRL_Sinpeng-2020}. Meanwhile, social media proved to be a crucial element in \textsc{Joseph Robinette “Joe” Biden Jr.}’s 2020 election win in the United States. Through an “influencer campaign”, Biden was able to reach out to young audiences on social media, particularly those of generation Z, which then translated to a massive voter turnout in that certain demographic \cite{CTX_Suciu-2020}.

In other elections, however, cases have occurred in which online support did not directly translate to election wins. One of the most prominent examples of the importance of social media in world politics is former United States (US) Vice President \textsc{Kamala Harris}’ social media campaign for the 2024 US presidential elections. Mainly targeting younger audiences through viral “memes” and other social media trends, Harris was able to amass widespread support for her campaign, having just over 5 million followers supporting her endeavors on TikTok and X (formerly Twitter) combined \cite{CTX_Lee-2024}. Similarly, the 2022 Philippine elections saw the Angat Buhay campaign of former Vice President \textsc{Leni Robredo}. Similarly to Harris, Robredo was able to garner the attention of young audiences on social media. Rallies in support of Robredo alongside Robredo’s track record as a politician made her a popular choice for millions as a capable presidential candidate \cite{CTX_Johnson-2022}.

Despite massive online support, both Harris and Robredo had lost their respective elections, the former only garnering 226 electoral votes (against \textsc{Donald Trump}’s 312 votes) and the latter garnering some 14.8 million votes (as opposed to \textsc{Ferdinand "Bongbong" Marcos, Jr}, who gathered 31.1 million) \cite{CTX_ABSCBN-2022,CTX_BBC-2024}. Given a possible disparity between social media popularity and election votes, the aim of this research is to provide a data-driven analysis on the effectiveness of social media as an indicator of election wins by observing social media trends at the time of both 2022 and 2024 elections, as well as comparing and contrasting these elections in terms of said trends.

Through \textsc{Natural Language Processing (NLP) Algorithms}, this paper aims to analyze the conversations on the aforementioned presidential candidates that had transpired in online spaces during pre-election seasons, namely X (formerly Twitter). Previous research endeavors have already shown the effectiveness of sentiment analysis in determining key themes behind social media posts, especially in the context of events such as elections. Thus, this research would like to push this idea further by not only contextualizing the data within a single setting. Rather, this paper aims to compare and contrast the election periods of the Philippines and US, given that, as already mentioned, the two countries experienced a supposed upset in terms of electoral candidate votes relative to their presence on social media.

This paper intends to provide a thorough a comparative study analysis between the 2022 Philippine presidential election campaigns of candidates Ferdinand “Bongbong” Marcos and Leni Robredo, with their respective running mates Sara Duterte and Francis “Kiko” Pangilinan, and the 2024 US presidential election campaigns of candidates Donald Trump and Kamala Harris, with their respective running mates James “JD” Vance and Timothy Walz. This research also aims to determine whether or not social media support directly translates to election success, or if other factors were present which had contributed to the losses of Harris and Robredo in their respective runs for presidency.