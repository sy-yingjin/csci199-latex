
\section{Context of Study}
In recent years, social media has had a major role as a platform through which political ideologies are spread and political discussions occur. This is especially apparent when observing the flow of recent elections in certain countries. During election seasons in the Philippines, renewed posts circulate social media characterising voters as ‘ill-informed’ and not understanding how democracy works for selling their votes to corrupt politicians \cite{CTX_Calimbahin-2023}. However, many may not realize that even the mere act of engaging with social media campaigns, even to criticize it, may actually be supporting the candidate.

When it comes to social media campaign regulations, the Philippines, along with Malaysia and Indonesia, are generally a laissez faire– not subjected to serious scrutiny or obligated to be transparent about campaign spending \cite{RRL_Tapsell-2020}. The 2016 Philippine presidential election is widely considered the first “social media election” for the Philippines, mainly due to how its winner, Rodrigo Roa Duterte, was able to utilize social media to establish a controversial image which mobilized his wide follower-base to rally in support of him, both online and offline \cite{RRL_Sinpeng-2020}. Tapsell’s [2020] analysis brings attention to an interview with Nic Gabunada, Duterte’s ‘campaign friend’, where he says Facebook became vital to their campaign after realizing that 45\% of Filipinos are on Facebook, via mobile phone, and that their goal for the campaign was to maximize awareness.

Cases, however, have occurred in which online support did not directly translate to election wins. The 2022 Philippine presidential elections saw the Angat Buhay campaign of former Vice President Leni Robredo garner the attention of young audiences on social media. Rallies in support of Robredo alongside Robredo’s track record as a politician made her a popular choice for millions as a capable presidential candidate \cite{CTX_Johnson-2022}. Despite massive online support, Robredo lost the elections, garnering some 14.8 million votes (as opposed to Ferdinand “Bongbong” Marcos, Jr, who gathered 31.1 million) \cite{CTX_ABSCBN-2022}. Given the possible disparity between social media popularity and election votes, the aim of this research is to provide a data-driven analysis on the effectiveness of social media as an indicator of election wins by observing social media trends at the time of both 2022 and 2024 elections, as well as comparing and contrasting these elections in terms of said trends.

Through Natural Language Processing (NLP) Algorithms, this paper aims to analyze the conversations on the aforementioned presidential candidates that had transpired in online spaces during pre-election seasons, namely X (formerly Twitter). Previous research endeavors have already shown the effectiveness of sentiment analysis in determining key themes behind social media posts, especially in the context of events such as elections. Thus, this research would like to push this idea further by not only contextualizing the data within a single setting. Rather, this paper aims to compare and contrast different election periods of the Philippines to create an Agent-based model (ABM) to, given the proper datasets, simulate and serve as a tool to predict future (Philippine) elections.

This paper intends to determine whether or not social media support directly translates to election success, or if other factors were present which had contributed to the losses of Robredo and worked to raise Duterte and Marcos Jr’s popularity.