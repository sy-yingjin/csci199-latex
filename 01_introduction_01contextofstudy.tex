\section{Context of Study}
In recent years, social media has had a major role as a platform through which political ideologies are spread and political discussions occur. This change, while gradual, makes it more convenient for data analysts to collect and analyze how people behave, specifically using Sentiment Analysis and Agent-based Modeling (ABM), and track the effectiveness and utility of social media as a campaign tool.

When it comes to social media campaign regulations, the Philippines,\\
along with Malaysia and Indonesia, are generally a laissez-faire and it has come to be a vital part of campaigns to maximize awareness on a candidate, as proven by the “first social media election”— the 2016 Philippine Presidential election \cite{Tapsell-2020, Sinpeng-2020}. Cases, however, have occurred in which online support did not directly translate to election wins. The 2022 Philippine presidential elections saw the Angat Buhay campaign of former Vice President Leni Robredo garner the attention of young audiences on social media. Rallies in support of Robredo alongside Robredo’s track record as a politician made her a popular choice for millions as a capable presidential candidate \cite{Johnson-2022}. Despite massive online support, Robredo lost the elections, garnering some 14.8 million votes (as opposed to Ferdinand “Bongbong” Marcos, Jr, who gathered 31.1 million) \cite{ABSCBN-2022}.

Given the possible disparity between social media popularity and election votes, the aim of this research is to provide a data-driven analysis on the effectiveness of social media as an indicator of election wins by observing social media trends at the time of both 2022 and 2024 elections, as well as comparing and contrasting these elections in terms of said trends.

Currently, many newer transformer models and variations of BERT, a large language model initially developed by Google, have been released, like\\
RoBERTa, DeBERTa and XLM-RoBERTa. When it comes to analyzing political science papers and studies, the newer models greatly outperform BERT, although XLM-RoBERTa shines more when it comes to cross-lingual applications \cite{Timoneda-2025}. This makes it an ideal transformer model to use when analyzing the political landscape on social media for a multilingual nation like the Philippines. Additionally, for a chaotic and constantly changing environment, Agent-based Modeling and Simulation (ABMS) excel at modeling systems full of autonomous, interacting agents, like social media users, that are able to change and adapt new behaviors \cite{Macal-2009}.

Through Natural Language Processing (NLP) Algorithms, this paper aims to analyze the conversations on the aforementioned presidential candidates that had transpired in online spaces during pre-election seasons, namely X (formerly Twitter). Previous research endeavors have already shown the effectiveness of sentiment analysis in determining key themes behind social media posts, especially in the context of events such as elections \cite{Prasanthi-2023}. Thus, this research would like to push this idea further by not only contextualizing the data within a single setting. Rather, this paper aims to compare and contrast different election periods of the Philippines to create an Agent-based model (ABM) to, given the proper datasets, simulate and serve as a tool to predict future (Philippine) elections.

This paper intends to determine whether or not social media support directly translates to election success, or if other factors were present which had contributed to the losses of Robredo and worked to raise Duterte and Marcos Jr’s popularity.
