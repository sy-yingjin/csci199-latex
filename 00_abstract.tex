\begin{thesisabstract}
Recent election periods have seen the rise of social media platforms as tools through which political ideologies are spread, and support for candidates are garnered. This increase of reliance on social media for political interaction allows for data analysts to collect, follow and study human behavior online during election periods. With developed masked language models (MLMs) and modeling tools, researchers are able to track and categorize sentiments expressed in social media posts.

The main motivation for this research is to determine the effectiveness of social media as an indicator of electoral wins. Through multilingual MLMs like XLM-RoBERTa and Agent-based modeling (ABM), this paper aims to perform sentiment analysis on posts from X, a popular social media platform for public discussions on politics in the Philippines, to draw insights on how citizens perceived the electoral candidates during pre- and proper election periods and see if it can be used as an indicator of an election win to serve as a tool for predicting future elections.  Posts related to the 2019 and 2025 midterm and 2022 presidential elections will be collected, annotated, and used to define an environment and agent for an Agent-Based model.
\end{thesisabstract}