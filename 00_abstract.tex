\begin{thesisabstract}
    In recent decades, social media has had a major role as a platform through which political ideologies are spread and political discussions occur. One of the most prominent examples of the importance of social media in world politics is former United States (US) Vice President Kamala Harris’ social media campaign for the 2024 US presidential elections. Mainly targeting younger audiences through viral “memes” and other social media trends, Harris was able to amass widespread support for her campaign, having just over 5 million followers supporting her endeavors on TikTok and X (formerly Twitter) combined.\cite{CTX_Lee-2024} Similarly, the 2022 Philippine elections saw the Angat Buhay campaign of former Vice President Leni Robredo. Similarly to Harris, Robredo was able to garner the attention of young audiences on social media. Rallies in support of Robredo alongside Robredo’s track record as a politician made her a popular choice for millions as a capable presidential candidate.\cite{CTX_Johnson-2022}  

Despite massive online support, both Harris and Robredo had lost their respective elections, the former only garnering 226 electoral votes (against Donald Trump’s 312 votes) and the latter garnering some 14.8 million votes (as opposed to Ferdinand ‘Bongbong’ Marcos, Jr., who gathered 31.1 million).\cite{CTX_ABSCBN-2022,CTX_BBC-2024} Given a possible disparity between social media popularity and election votes, the aim of this research is to provide a data-driven analysis on the effectiveness of social media as an indicator of election wins by observing social media trends at the time of both 2022 and 2024 elections, as well as comparing and contrasting these elections in terms of said trends.
    
Through Natural Language Processing (NLP) techniques, the researchers aim to analyze social media data on the aforementioned presidential candidates that had transpired in online spaces during pre-election seasons, namely Facebook, TikTok, and X (formerly Twitter). ML models for sentiment analysis, such as BERT models, will be used to analyze these different posts to determine \\whether or not social media support directly translates to election success.
\end{thesisabstract}