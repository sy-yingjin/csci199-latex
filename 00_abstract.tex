\begin{thesisabstract}
Recent election periods have seen the rise of social media platforms as tools through which political ideologies are spread, and support for candidates are garnered. The most recent and popular example of which was the landslide victory of Rodrigo Roa Duterte during the 2016 Philippine presidential election, which many attributed to his social media campaigns. The same cannot be said for when former vice president Maria Leonor “Leni” Robredo, despite widespread support in online spaces, lost the 2022 Philippine presidential election to Ferdinand “Bongbong” Marcos Jr. 

The main motivation for this research is to determine the effectiveness of social media to predict electoral wins. Through the Robustly Optimized BERT Pre-trained Approach (RoBERTA), its multilingual variant (XLM-RoBERTa), and Agent-based modeling (ABM), this paper aims to perform sentiment analysis on posts from X (formerly Twitter), a popular social media platform when it comes to public discussions on politics in the Philippines, to draw insights on how citizens perceived the winning and first runner-up electoral candidates during pre- and proper election periods and see if it can be used as an indicator of an election win to serve as a tool for predicting future elections.
\end{thesisabstract}