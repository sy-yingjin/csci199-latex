\begin{thesisabstract}
Recent election periods have seen the rise of social media platforms as tools through which political ideologies are spread, and support for candidates are garnered. Despite widespread support in online spaces, however, former vice presidents Maria Leonor "Leni" Robredo and Kamala Devi Harris lost their respective presidential elections in 2022 (in the Philippines) and 2024 (in the United States) to Ferdinand “Bongbong” Marcos Jr. and Donald John Trump. The main motivation for this research is to determine the effectiveness of social media as an indicator of electoral wins. Through the Robustly Optimized BERT Pre-Trained Approach (RoBERTa) and other NLP techniques, this paper aims to perform sentiment analysis on posts from X (formerly Twitter), a popular social media platform when it comes to discussing politics for both countries, to draw insights on how citizens perceived both the winning and first runner-up electoral candidates during pre- and proper election periods and compare these perceptions to the results of each presidential election.
\end{thesisabstract}