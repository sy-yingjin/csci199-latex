\begin{thesisabstract}
Recent election periods have seen the rise of social media platforms as tools through which political ideologies are spread, and support for candidates are garnered. The most recent and popular example of which was the landslide victory of Rodrigo Roa Duterte during the 2016 Philippine presidential election, which many attributed to his social media campaigns. This increase of reliance on social media for political interaction allows for data analysts to follow and study human behavior online during election periods. With developed transformer models and modeling tools, researchers are able to track and categorize sentiments expressed in social media posts.

The main motivation for this research is to determine the effectiveness of social media as an indicator of electoral wins. Through the Robustly Optimized BERT Pre-trained Approach (RoBERTA), its multilingual variant (XLM-RoBERTa), and Agent-based modeling (ABM), this paper aims to perform sentiment analysis on posts from X, a popular social media platform when it comes to public discussions on politics in the Philippines, to draw insights on how citizens perceived the  electoral candidates during pre- and proper election periods and see if it can be used as an indicator of an election win to serve as a tool for predicting future elections.
\end{thesisabstract}